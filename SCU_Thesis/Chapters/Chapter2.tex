%!TEX root = ../MainBody.tex

% 第二章
\chapter{相关技术}% 使用\cite{}命令引用数据库中文献
本章介绍在该项目开发中使用到的主要技术。包括 Android, Spring Boot, Spring JPA, Spring Security + JWT, 
Restful, Json, Mysql , 并使用 Maven 和 Gradle 作为依赖管理工具。

\section{Android系统}
Android\cite{Android},是基于 Linux 内核的由 Google 领导和开发的移动设备操作系统,主要为触摸移动设备例如智能
手机和平板设计的。除此之外,还广泛应用在智能电视,智能手表,无线路由,智能冰箱,智能汽车等设备
上。Android 主要运行在 ARM 处理器架构的设备上,也有少量 X86 架构的 Android 设备。

目前 Android 的市场占有率高达 85.9\%。2008 年 9 月 23 日,第一步 Android 手机 HTC Dream 发布。从最初的 Android 1.0,1.1,
 1.5 逐渐到2.0,3.0, 到目前最新的 Android Pie 9.0,API 级别为 28,在 10 年时间内,Android 经过了十余
 次的迭代更新。核心的 Android 源代码以 Android 开源项目(AOSP)而著名。2017年,Android设备超越 Microsoft Windows,成为全球第一大操作系统。

\begin{itemize}
    \item Android Activity 组件
    
    Activity\cite{Activity} 相似于网页的一个页面,不过它是安卓操作系统上独有的页面。它是通过 xml 的方式描述
    安卓的页面布局。由于移动设备的独特性,所以 Android 系统的屏幕一次只显示一个 Activity,不过在 Activity 内
    可以包含多个 Fragment 片段。Fragment 类似于 Activity,但它不能单独显示在 Android 设备页面,而是只能包含
    在 Activity 内,同样有 xml 文件定义的布局。

\end{itemize}



 \section{Spring框架}
Spring,即 Spring Framework\cite{SpringFramework},是 Java 平台的控制反转容器的应用框架。它是一个开源项目,作为 Enterprise
JavaBean 模型的替代方案。在诞生之初,Spring 的主要目的是用来替代更加重量级的企业级 Java 技术\cite{SpringInAction},
Spring 为企业级 Java 开发提供了一种相对简单的方法,通过依赖注入和面向切面编程,用简单的 Java 对象
实现 EJB 的功能。

Spring 框架包含了一些模块用于提供服务\cite{WikiSpring} 包括 Spring 核心容器,面向切面编程,认证和授权,数据访问,控制反转容器,消息,模型-视图-控制器,事务管理等。
% \begin{itemize}
%     \item   Spring核心容器: 这是Spring的基础模块,提供spring 容器(BeanFactory 和 ApplicationContext)。
%     \item   面向切面编程:允许实现横切的特点。
%     \item   认证和授权:配置安全过程,通过 Spring Security子项目支持很多标准,协议和工具。
%     \item   数据访问:使用java平台的JDBC,关系数据库管理系统,对象关系映射工具,与非关系型数据库。
%     \item   控制反转容器: 配置应用组件和Java对象的生活周期管理,主要通过依赖注入完成。
%     \item   消息:通过Java消息服务(JMS)的方式配置注册消息队列的消息侦听器,通过标准的JMS API 提升发送消息。
%     \item   模型-视图-控制器:基于 HTTP 和 servlet 的框架提供面向 web 应用和 RESTful Web 服务的自定义和扩展。
%     \item   远程访问框架:可配置的远程程序调用风格的通过网络支持 Java 远程方法调用的Java对象传输。和包括 Web 服务的
% 基于HTTP的协议。
%     \item   事务管理:统一了事务管理API与Java对象的协作事务。
% \end{itemize}

\section{Spring Boot框架}
虽然 Spring 的组件代码是轻量级的,但它的配置却是重量级的 \cite{SpringBootInAction} 。
 Spring Boot \cite{SpringBoot} 将很多魔法带入 Spring 应用程序的开发之中,包括下面四个核心方面:

\begin{itemize}
    \item    自动配置:针对很多 Spring 应用程序常见的应用功能,Spring Boot 能提供相关配置。
    \item    起步依赖:告诉 Spring Boot 需要什么功能,它就能引入需要的库。
    \item    命令行界面:这是 Spring Boot 的可选特性,借此你需要代码就能完成完整的应用程序,无需传统项目构建。   
    \item    Actuator: 让你能够深入运行中的 Spring Boot 应用程序,一探究竟\cite{SpringBootInAction}。
\end{itemize}

\section{Spring Data JPA技术}
JPA 是 Java Persistence API 的简称。是 Spring 集成的数据持久化框架。目前市场上的 ORM(object/relational metadata)框架由 Mybatis,
Hibernate,Spring Data JPA\cite{SpringDataJPA2}. Spring Data JPA 是一种使用相对简单的方式。

JPA 包含三方面的内容:API 标准,面向对象的查询语言,ORM 元数据的映射。JPA 的宗旨是为 POJO 提供持久化标准
规范\cite{SpringDataJPA}。

Spring Data JPA 大大简化了我们 DAO(Data access object)层的开发,使我们只要创建普通的 Java 对象,
即pojo,再加上标记,就可以通过 JPA 的 API 操作数据库。

\section{Spring Security安全框架}
Spring Security\cite{SpringSecurity} 是一个强大的高度可定制的认证和入口控制框架。它是事实上的保护基于 Spring 的应用的标准。
Spring Security 是专注于 Java 应用的认证和授权框架。Spring Security 最初在2003年后半年开始于 Acegi Security, 
由 Ben Alex 开发,2004年在 Apache 证书下发布。接下来,变成了官方的 Spring 的一个子项目。

Spring Security的特点:

\begin{itemize}
    \item 对于认证和授权来说都是易于理解和可扩展的。
    \item 保护应用免受会话攻击,点击劫持,扩展请求伪造等攻击。
    \item 继承了 Servlet API。
    \item 可选的与 Spring Web MVC 集成。
\end{itemize}

\section{Maven构建工具}
Maven 是 Apache 基金会下的一个项目,即 Apache Maven。官方定义\cite{Maven}: Apache Maven 
是一个软件项目管理和理解工具。基于项目对象模型(POM)的概念,Maven 可以从一个中心消息
片管理一个项目的构建,报告和文档。维基百科定义\cite{WikiMaven}:
 Maven是一个主要用于 Java 项目的自动工具。

% Maven定位构建软件的两个方面:第一,它描述如何构建软件,第二,它描述依赖。不像早期的Apache Ant,它使用惯例构建软件,
% 只有异常需要写下来。一个XML文件描述项目如何构建,它对外部模块和组件的依赖,构建顺序,目录,需要的插件。它天生就是来处理
% 定义好的目标,执行特定的定义好的任务例如编译代码和打包。

Maven 从一个或者多个仓库例如 Maven 2 中心仓库动态地下载 Java 库和 Maven 插件并存储到本地缓存中。 本地下载地库可以被本地项目更新,
公共仓库同样可以更新。Maven 同样可以用来构建和管理其他语言的项目例如 C\#,Ruby,Scala,或者其他语言。

\section{Gradle构建工具}
Gradle\cite{Gradle} 是一个开源自动构建系统,在 Apache Ant 和 Apache Maven 的
概念之上,引入了基于 Groovy 的域定义语言(DSL)来替代 Maven 使用的用来声明项目配置的 XML 形式。
Gradle 使用有向无环图("DAG")来决定任务的执行顺序,Gradle 也会缓存任务的输出,当 Gradle 发现任务的输入输出在构建运行时改变,这个任务将会再次执行。

% Gradle使用有向无环图("DAG")来决定任务的执行顺序。Gradle 被设计为多项目构建的。它支持通过智能的决定构建树的
% 哪部分过期增量构建;任何任务独立的部分不需要再次执行。Gradle 检查构建的输入输出来查看自上次构建调用后一个任务的输出是否
% 改变,如果没有,这个任务被认为是最新的,将不会被执行。Gradle 也会将任务配置作为输入的一部分。Gradle 允许并行的任务执行和通过一个 Worker API来进行内部任务,并行化
% 导致了更快的性能。Gradle 也会并行下载依赖元数据。

% Gradle 也会缓存任务的输出。如果一个任务已经在另一台计算机执行过,Gradle 会跳过本地执行,从构建缓存中获取任务输出。
% 这种典型的用例让CI构建推送到一个共享的任务缓冲区,并且允许开发者拉取它。本地构建缓存同样能够被使用来重用以前在相同计算机
% 上构建的输出。

% 当Gradle发现任务的输入输出在构建运行时改变,这个任务将会再次执行。这个任务将使用增量的API来知道具体哪些文件发生了改变。
% 有了这个消息后,这个任务不需要重新构建所有。

\section{Mysql数据库}
MYSQL\cite{WikiMysql}是一个开源关系型数据库管理系统。

Mysql 是常用的开源数据库,被很多著名的网站使用,包括 Google,Facebook,Twitter,
Flickr,YouTube。Mysql 是用 C 和 C++ 写的, 支持很多平台,包括 Windows,MacOS,Unix,以及 Linux 的多个
发行版,你甚至可以自行下载它的代码来编译。它支持存储百万级的数据存储。不仅如此,Mysql 还参考了 SQL 标准,并对 SQL 
提供了很好的支持。它被设计由独立模块组成的多层服务器设计,被设计为完全多现成的,并且提供了事务和非事务的存储
引擎供你选择。它在 MyISAM 存储引擎中使用非常快的 B 树索引。Mysql 支持大的数据库,官方曾经使用 MySQL 服务器
包含了 5000 万条记录。为多种编程语言提供了 API 接口\cite{MysqlMainFeature}。

Mysql 被设计为 Server 和 存储引擎分离的,并且存储引擎是可插拔的,你可以从 MySQL 服务器中载入和卸载存储引擎。
不同存储引擎提供了一些不同的特点,默认的存储引擎是 InnoDB。Mysql 支持的存储引擎有 InnoDB,MyISAM,Memory,
CSV,Archive,Blackhole,NDB,Merge,Federated,Example 等。这些引擎之间的不同在于提供了不同的特点,让你根据
你的数据特点选择。主要不同在以下方面: B 树索引的支持,备份点实时恢复的支持,簇数据库的支持,簇索引,是否压缩数据,
是否缓存数据,是否加密数据,是否支持外键,是否支持全文索引,是否支持地理空间数据类型,是否支持地理空间索引,是否支持哈希
索引,是否支持缓存索引,支持的锁粒度,MVCC 的支持,赋值的支持,存储容量限制,T 树索引的支持,事务的支持,数据字典的更新等\cite{MysqlNewFeature}。

\section{Restful原则}
REST ( Representational State Transfer),表达性状态传输, 是一种创建规模网络服务的软件架构方法。
术语 REST 由 Roy Field 在他的博士论文中提出,解决了很多原则问题。包括了
以下原则\cite{RESTfulService}:

\begin{itemize}

\item Uniform interface

REST 的核心是资源,资源由统一资源定位符(Uniform Resource Identifiers , URIs)标识,资源与它们的
表示(它们提供给客户端的格式)分离。REST 没有具体要求格式,但通常包括 XML 和 JSON。

除此之外,资源表述是自我描述性的,更具体地说,这意味着成功的回复处理必须返回足够充分的信息。

REST 的另一个不同的属性是客户端完全是通过超媒体交互的,由应用服务器动态提供,除了端点,客户端对如何与 RESTful
服务交互一无所知。这种约束被称为应用状态引擎超媒体 ( Hypermedia as the Engine of Application State , HATEOAS )\cite{RESTfulService}。

\item Client-Server模型

客户端服务器模型表示 REST 接受客户端关注点的分离,例如用户交互或者用户状态管理,与服务器关心的
存储和规模性。

这种解耦确保,提供一个达成一致的接口,客户端和服务器能够独立的开发。它同样减少了复杂性,提升了性能调整
的效率。

\item 无状态

REST提倡无状态。服务器不保存客户端的状态。所有需要执行操作的请求信息都包含在请求中(作为URL,请求
体,HTTP头部的一部分)。

\item 可缓存的

RESTful web服务必须提供缓存能力。服务器能够暗示如何和缓存回复多久。客户端可以使用缓存的回复而不是联系服务器。

\item 层级系统

这种客户端和服务器的沟通风格,客户端并不清楚它们正与哪台具体的服务器交互。这种属性允许引入中间
服务器,例如处理安全或者提供负载均衡能力的服务器。

\item 按需代码

即使是REST架构的一部分,这个原则也是可选的。服务器可以临时的扩展客户端的功能通过传输可执行的代码。
例如可以传输给基于web的客户端 JavaScript 代码来自定义功能。

\end{itemize}
\section{JSON数据格式}
JSON, JavaScript Object Notation\cite{WikiJSON}. JavaScript 对象符号,是一种人类易读的文本 , 用来传输数据对象
包含属性-值键值对和数组数据类型的文件格式。它是一种非常常见的以部的浏览器-客户端沟通的数据格式。

JSON是语言独立的书写格式,衍生自 JavaScript。官方的 Internet 媒体类型是
application/Json。JSON 文件名使用后缀.json。

\section{JWT身份认证}
JWT(JSON Web Token)\cite{JWT}是一个开源标准(\href{https://tools.ietf.org/html/rfc7519}{RFC 7519}),定义了
在团体之间使用JSON对象安全传输信息的切确和自我包含方式。这条信息可以被验证,因为它是数字签名的。JWT 可以使用
密匙签名(使用HMAC算法)或者公共/私有密匙对使用 RSA 或者 ECDSA。

JWT 被用来授权是最常见的使用场景。一旦用户登录,每个后续请求都包含 JWT,允许用户获取路由,服务,和 token 许可的资源。
如今使用 JWT 的单点登录特点被广泛使用,因为它的小的负载头部和在不同域间轻松使用的能力。




