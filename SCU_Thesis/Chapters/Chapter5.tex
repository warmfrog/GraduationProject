%!TEX root = ../MainBody.tex

% 第5章
\chapter{系统实现}
整套系统使用 Mysql 作为数据库,以 Spring Boot 内置的 Tomcat 作为服务器,使用 Spring JPA 进行数据库
操作,Android 手机作为客户端,使用了前后端分离的开发方式。

\section{程序开发工具}

后台开发: 使用 Jerbain 公司的 IDEA 作为主要开发工具,使用 Java 8 作为主要开发语言,并安装了 Mysql 作为数
据库, 使用 Maven 作为依赖管理工具。暴露Restful服务接口。使用了DTO数据传输对象模式。

Android 开发: 使用Android Studio 作为主要开发工具,并使用 API 27.0.1 平台,同样以 Java 8 作为主要开发语言,
使用 Google 官方推荐的 Gradle 作为依赖管理构建工具。使用一台运行 Android 8.0 的真机作为调试用机。

前后端共同调试时,需要手机访问到服务器端,因此要求手机和服务器在同一个网络下。这里使用手机共享热点给电脑
的方式,保证了调试时手机和电脑在同一个局域网下,并且能够互相访问。

\section{后台实现}

\subsection{消息响应格式}

\begin{verbatim}
//封装的消息响应类
public class ApiResponse implements Serializable {
    //代表服务器处理的状态信息
    Status status;
    //代表服务器返回的数据
    Object data;
    //代表错误代码和错误信息
    ApiError error;
    //代表返回结果总数,可选
    Long totalItems;
    //代码返回的总页数,可选。
    Integer totalPages;

    public enum Status implements Serializable {
        ok(0),
        error(1);

        Integer value;

        Status(Integer value) {
            this.value = value;
        }
    }

    public static class ApiError implements Serializable {
        Integer errorCode;
        String errorMessage;
    }
}
\end{verbatim}

后端返回数据时,统一封装为该格式后转换为 Json 数据返回。

status表示处理状态: ok表示处理正常,error表示
处理错误。

data表示返回的具体数据。

error表示出错时的错误代码和错误提示信息。

totoalItems表示返回的对象数。

totalPages 表示一共有多少页数据。

\subsection{JWT用户认证实现}
使用JWT认证首先要导入maven依赖

\begin{lstlisting}
    <dependency>
            <groupId>io.jsonwebtoken</groupId>
            <artifactId>jjwt</artifactId>
            <version>0.9.1</version>
        </dependency>
\end{lstlisting}

配置Spring Security, 对所有请求添加JWT的过滤器,一个登录过滤器,一个认证过滤器。

\begin{verbatim}
    protected void configure(HttpSecurity http)
     throws Exception {
        http.cors().and().csrf().disable().authorizeRequests()
                //添加登录过滤器
                .addFilter(
                    new JWTLoginFilter(authenticationManager()))
                //添加认证过滤器
                .addFilter(
                    new JwtAuthenticationFilter(
                        authenticationManager()));
    }
\end{verbatim}

\subsection{登录过滤器实现}
\begin{verbatim}
    public Authentication attemptAuthentication(
        HttpServletRequest request, 
        HttpServletResponse response) 
        throws AuthenticationException {
        try {
            //获取请求中的用户信息
            User user = new ObjectMapper().readValue(
                request.getInputStream(), User.class);
            //将用户信息与数据库比对
            return authenticationManager.authenticate(
                    new UsernamePasswordAuthenticationToken(
                            user.getUsername(),
                            user.getPassword(),
                            new ArrayList<>()
                    ));
        } catch (IOException e) {
            e.printStackTrace();
        }
        return null;
    }

    //如果认证成功
    protected void successfulAuthentication(
        HttpServletRequest request, 
        HttpServletResponse response,
        FilterChain chain,
        Authentication authResult)throws 
            IOException,
            ServletException {
        //如果认证成功,则为该用户生成一个token,设置签名,并设置过期时间
        String token = Jwts.builder()
                .setSubject(((
                    org.springFramework.security.core.userdetails.User) 
                authResult.getPrincipal()).getUsername())
                .setExpiration(new Date(
                    System.currentTimeMillis() + 60 * 60 *24 * 1000))
                .signWith(SignatureAlgorithm.HS512, "sharebook")
                .compact();
        //将token返回到请求头部
        response.addHeader("Authorization", "Bearer" + token);

    }
\end{verbatim}

\subsection{认证过滤器}
\begin{verbatim}
    protected void doFilterInternal(
        HttpServletRequest request,
        HttpServletResponse response, 
        FilterChain chain) throws IOException, ServletException {
        //获取用户请求头的认证信息
        String header = request.getHeader("Authorization");
        if(header == null || !header.startsWith("Bearer")){
            chain.doFilter(request,response);
            return;
        }
        //获取用户token
        UsernamePasswordAuthenticationToken authenticationToken = 
            getAuthentication(request);
        //对token认证。
        SecurityContextHolder.
            getContext().
            setAuthentication(authenticationToken);
        chain.doFilter(request,response);
    }

    private UsernamePasswordAuthenticationToken 
    getAuthentication(HttpServletRequest request){
        //获取请求头
        String token = request.getHeader("Authorization");
        if(token != null){
            String user = Jwts.parser()
                    .setSigningKey("sharebook")
                    .parseClaimsJws(token.replace("Bearer", ""))
                    .getBody()
                    .getSubject();

            if(user != null){
                return new UsernamePasswordAuthenticationToken(
                    user, null, new ArrayList<>());
            }
            return null;
        }
        return null;
    }
\end{verbatim}

\section{Android客户端主要实现}
Android 客户端实现包括了 UI 代码布局定义 XML 文件及相应的 Java类,一些资源文件如图片,字符串常量
,颜色定义,菜单,尺寸定义,style 等,除了二进制文件外,其他资源 都以 XML 格式定义,代码除了处理
 UI 片段的代码,还有 一些异步任务类,数据持久化 Dao类,网络请求类等。

 Android 客户端的界面都是由 XML 文件定义的 视图,并由 Java 类加载相应的视图 XML 定义文件。界面主要
 分为 Activity 和 Fragment。 Activity 是单独的页面,及 Android 客户端上的一个屏幕, 而 Fragment 则
 包含在 Activity 布局中。

 Android 客户端不能在 UI 主线程进行操作数据库,网络等操作,所以在 Activity 或 Fragment 内部定义了一些
 内部异步任务类,继承于 AsycnTasks, 用来执行操作数据库,网络请求等耗时的会阻塞 UI 的操作,并在获取到
 结果时,修改 UI 的内容或结果。

 Android 客户端包含了一个本地的数据库 Sqlite\cite{Sqlite},主要通过 Android 的 Room 组件 来操作数据库。Room持久性库\cite{Room}是
 Google 官方提供的一个 Android 客户端的数据持久化库,它在 Sqlite 之上提供了一个抽象层,用来简化对数据库
 的操做。本地数据库主要用来缓存从网络请求中获得的结果。用来避免重复的网络请求,避免浪费网络资源和服务器资源。
 当用户打开客户端时,客户端从网络加载数据,将首先缓存在本地数据库中,然后从本地数据库加载数据,显示到 UI 界面。
 当用户下拉时,如果本地数据库数据加载完,将出发一个回调,从服务器加载之后的数据,同样的首先保存到数据库。

 Android客户端的网络请求使用了 Google 推荐的 Retrofit 库。Retrofit\cite{Retrofit} 是一个 Android 和 Java 的 HTTP 客户端。
 它可以将 URL 请求地址与方法绑定,加上一些标记,就可以轻松的进行网络的请求,大大简便了网络请求的开发。通常调用
 方法需要实现两个回调函数,一个成功的回调和失败的回调。Retrofit 提供了顺序实行和异步执行的方式。如果需要顺序
 执行,调用execute方法即可;如果需要异步执行,则调用 enqueue 方法将请求添加到请求队列即可。

\subsection{主界面实现}
主界面是一个 MainActivity,包含一个 Fragment 容器和底部的导航栏,默认打开主界面,Fragment 容器包含的
是 MainFragment。 底部导航栏包含一组菜单,当点击底部的任一个选项,当前 Fragment 容器包含的 Fragment
都会被压入 MainActivity 的 Fragment 栈中,并创建选项对应的 Fragment 页面布局, 替代 Fragment 容器中
的 Fragment页面。这些界面 Fragment 是主界面 MainFragment,发布项选择 mydialog, 消息界面 message\_list 和 
我的界面 Fragment\_self\_intro, 主界面布局代码见下方。

\begin{verbatim}
    <?xml version="1.0" encoding="utf-8"?>
<android.support.constraint.ConstraintLayout
    tools:context=".ui.activity.MainActivity">

    <!-- Fragment 容器 -->
<FrameLayout
    android:id="@+id/Fragment_container"/>

<!-- 底部导航栏-->
    <android.support.design.widget.BottomNavigationView
        app:menu="@menu/navigation" />
</android.support.constraint.ConstraintLayout>
\end{verbatim}

\subsection{登录实现}
Android客户端在成功登录后,会获取到服务器返回的 token 信息,Android 客户端以后的每个请求都需要将 token 添加
到网络请求头中,这通过给 retrofit 的 client 添加一个过滤器来实现。在过滤器中拦截每一个请求,对每个请求,添加
本地保存的 token 到认证头部。下方是实现:

\begin{verbatim}
    //创建一个过滤器
    Interceptor mTokenInterceptor = new Interceptor() {
            @Override
            public Response intercept(Chain chain) throws IOException {
                //获取用户请求
                Request orginalRequest = chain.request();
                SharedPreferences userConfig = getApplicationContext()
                .getSharedPreferences("UserConfig", MODE_PRIVATE);
                //从本地获取 token信息
                String token = userConfig.getString("token", null);
                username = userConfig.getString("current_user", null);

                if (token == null) {
                    return chain.proceed(orginalRequest);
                }
                //将 token 添加到认证头部
                Request request = orginalRequest.newBuilder().
                header("Authorization", token).build();
                return chain.proceed(request);
            }
        };
        //根据过滤器创建客户端
        client = new OkHttpClient.Builder()
        .addInterceptor(mTokenInterceptor).build();
        //根据客户端创建 Retrofit 实例。
        retrofit = new Retrofit.Builder()
                  .baseUrl("http://192.168.42.44:8081")
                .addConverterFactory(JacksonConverterFactory.create())
                .client(client)
                .build();
\end{verbatim}














