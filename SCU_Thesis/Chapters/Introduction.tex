%!TEX root = ../MainBody.tex

% 第一章
\chapter{绪论}
本章简单介绍了共享经济的发展背景,我国图书销售和流通情况,以及关于共享图书的国内外研究现状。

\section{项目背景与意义}
    如今,4G 技术已经非常成熟并且普及,5G\cite{Wiki5G} 技术将在未来两年内逐渐普及。移动通信基础设施的完善,
    为移动互联网的发展奠定了坚实的基础。共享经济就是在移动互联网非常发达的基础上发展起来的。
    共享单车,共享汽车,共享充电宝,共享雨伞等一系列共享经济在全球发展火热。

    共享经济主要来源于制度经济学的两大理论:交易成本理论和产权理论\cite{share2018}。 共享经济本质是提高稀缺资源, 
    闲置资源的利用率\cite{SharingEconomy},在资源的生命周期内充分使用资源的价值。共享经济有多种商业模式,
    但共同的特征是其商业的重点是使用权而不是所有权\cite{theory}。共享性需求的产生和发展是经济增长和社会发展的必然结果\cite{think}。共享经济借助网络等第三方平台,将供给方的闲置资源使用权暂时性转移,
    实现生产要素的社会化,通过提高存量资产的使用效率为需求方创造价值,促进社会经济的可持续发展\cite{reason}。实质是使交易成本最小化\cite{trade}。
    可见,共享经济作为一种崭新的经济模式有着无限的潜力。可以举一反三,横向扩展到其他领域,
    比如说共享图书。

    据统计\cite{BookSales},2018 年中国图书销售规模达 894 亿。这表明虽然电子阅读虽然规模不断增长,但纸质图书仍然有很大规模。纸质书
    一个问题是它的利用率并不高,除了学辅教材,专业书等,其他的书籍,我们很少会阅读第二遍。如果这些书籍能够在人与人之间
    相互流通,那不仅提高了图书的利用率,而且有助于保护环境,降低国民的阅读门槛,提升国民的阅读率。
    本文将尝试设计并实现一个共享图书系统,提供一个用户之间共享二手书的平台。通过该平台,将图书真正的在读者之间流动起来。

\section{国内外研究现状}

    2010年,随着 Uber, Airbnb 等共享平台的出现,共享经济作为一种新的经济模式开始发展起来\cite{Airbnb}。国内共享单车共
    享汽车发展的最为火热,然而共享图书却并没有那样流行。根据调查,国内共享图书平台借书人、摩布图书、Book++、
    约书、亿屏借书等发展前景并不乐观,普遍未实现盈利,有的甚至服务已经无法访问。借书人以微信小程序的形式,用户使
    用小程序发布借阅图书,并支付服务费和押金,目前尚未实现盈利。摩布图书\cite{mobu}则是以在广州,成都等地投放共享书柜的形式,
    以租金和会员费,主要以儿童图书为主。至于其它三家服务或网站发现已经无法访问。闲鱼\cite{2taobao}是一个二手物品交易平台,
    它上面的二手书交易量非常可观,但它并不是一个专一面向二手书的交易平台。

\section{研究内容}

本文主要研究设计和开发一套移动端共享图书系统,包括后台和 Android 移动客户端设计开发等工作内容。
 Android 移动端的开发,包括 UI 的设计开发,基本功能的开发实现等,主要实现以下内容:实现基本用户
的注册登录功能;实现基本图书的发布功能,使用户仅仅扫描图书条形码或者手动输入条形码,就能够获
取到所有的图书相关信息,再补充一些发布信息就能够完成图书的发布;实现用户之间的沟通交流。通过该平台,直接在移动端完成
图书的租赁,购买,下单,支付,确认收货,沟通交流等功能。使得用户能够以较低门槛即可共享和租赁图书。
共享图书后台开发的主要内容包括数据实体设计,Restful 服务接口设计,用户的身份认证,根据网
络API获取图书信息等内容。

\section{特色}

\begin{itemize}
    \item 一个类似于 P2P 的用户与用户之间的二手书交易平台。交易过程中完全无第三方的参与。
    \item 移动端的二手书交易平台,用户随时随地都可以互联交易二手书。
    \item 使用简单,低门槛。通过扫描图书条形码即可简单地上传二手书。
    \item 图书交付方式自由,用户之间通过沟通交流,自由决定交付时间地点。
\end{itemize}


\section{论文组织结构}

\begin{itemize}
    \item 第一章,绪论。概括了项目研究背景和目前国内外的研究现状。
    \item 第二章,相关技术。介绍了本项目设计的相关技术的简介。
    \item 第三章,需求分析。分析了共享图书系统的功能需求。
    \item 第四章,总体设计。详细描述了本项目的服务端接口设计,数据库设计,客户端设计。
    \item 第五章,系统实现。详细介绍了服务端的用户登录认证实现和 Android 客户端实现。
\end{itemize}
